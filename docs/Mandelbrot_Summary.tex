\documentclass[12pt,a4paper]{article}
\usepackage[utf8]{inputenc}
\usepackage{amsmath}
\usepackage{graphicx}
\usepackage{geometry}
\geometry{margin=1in}

\begin{document}

\begin{center}
  {\LARGE \textbf{Visualizing the Mandelbrot Set: A GPU-Accelerated Approach}} \\[0.5cm]
  {\large Rajat Dahal \quad Falano Dhiskino} \\[0.2cm]
  {\small December 31, 2025}
\end{center}

\section*{Introduction}
The project explores the Mandelbrot set using real-time rendering techniques. The Mandelbrot set consists of complex numbers \(c\) for which the iteration \(z_{n+1}=z_n^2+c\) (starting at \(z_0=0\)) does not diverge. By leveraging GPU fragment shaders we achieve smooth zooming and interactive exploration that is impractical with CPU-bound rendering.

\section*{Mathematical Foundation}
Use the escape-time algorithm per pixel:
\begin{enumerate}
  \item Map pixel coordinates to a complex point \(c\).
  \item Iterate \(z_{n+1} = z_n^2 + c\) up to a maximum iteration count.
  \item If \(|z|>2\) before the max iterations, record the escape iteration for coloring; otherwise treat the point as inside the set.
\end{enumerate}

\section*{Tools and Technologies}
\begin{itemize}
  \item Primary Framework: p5.js
  \item Shading Language: GLSL (fragment shaders for escape-time computation)
\end{itemize}

\section*{Conclusion}
Offloading iterative calculations to the GPU provides a large performance gain, enabling real-time interaction and high-resolution rendering. This work demonstrates coordinate mapping, shader pipelines, and practical GPU-accelerated visualization techniques.

\end{document}
