\documentclass[12pt,a4paper]{article}
\usepackage[utf8]{inputenc}
\usepackage{graphicx}
\usepackage{amsmath}
\usepackage{amsfonts}
\usepackage{amssymb}
\usepackage{fancyhdr}
\usepackage{titlesec}
\usepackage{enumitem}
\usepackage{geometry}
\geometry{a4paper, margin=1in}

% Title formatting
\titleformat{\section}[block]{\large\bfseries\centering}{}{0em}{}
\titleformat{\subsection}[block]{\normalsize\bfseries}{}{0em}{}

\begin{document}

% Title page
\begin{center}
    {\LARGE \textbf{Kathmandu University}} \\[0.3cm]
    {\large Department of Computer Science and Engineering} \\[0.2cm]
    {\normalsize Dhulikhel, Kavre} \\[1.5cm]
    \begin{figure}[h]
        \centering
        \includegraphics[width=0.3\textwidth]{logoku.png}
        \label{kulogo}
    \end{figure}\\[1cm]
    {\huge \textbf{Project Concept Note}} \\[0.5cm]
    {\large [Code No: COMP 342]} \\[2cm]
    
    \begin{minipage}[t]{0.45\textwidth}
        \textbf{Submitted by:} \\[0.5cm]
        \textbf{Masto Christmas Bhandari (Roll No: 08)} \\
        \textbf{Utsav Adhikei (Roll No: 03)} \\
        Department of Computer Science and Engineering \\
        Kathmandu University
    \end{minipage}
    \hfill
    \begin{minipage}[t]{0.45\textwidth}
        \textbf{Submitted to:} \\[0.5cm]
        \textbf{Mr. Dhiraj Shrestha} \\
        Assistant Professor \\
        Department of Computer Science and Engineering \\
        Kathmandu University
    \end{minipage}
\end{center}

\newpage

\title{Typescape: A 2D Typing Challenge Game with Progressive Levels}
\author{Masto Christmas Bhandari and Utsav Adhikei}
\date{\today}
\maketitle
\begin{center}
    {\Large \textbf{Concept Note}} \\
    {\large Typescape — A 2D Typing Challenge} \\
    \vspace{0.5cm}
\end{center}

\section*{Introduction}
Typescape is a compact 2D typing game where players advance by correctly typing words shown on screen. The project focuses on applying practical graphics techniques to build a polished, responsive demo with clear audio-visual feedback.

\section*{Objectives}
\begin{itemize}[nosep]
    \item Implement a tight typing-to-move interaction loop with instant feedback.
    \item Create and animate pixel-art sprites and tilemap levels.
    \item Use camera follow, parallax layers, and simple particle effects for polish.
    \item Optimize rendering with texture atlases and minimal overdraw for smooth performance.
    \item Document the asset pipeline and build steps for easy reproduction.
\end{itemize}

\section*{Key Features}
\begin{itemize}[nosep]
    \item Four progressive levels with increasing word difficulty and pacing.
    \item Immediate visual/audio cues for correct/incorrect typing.
    \item Leaderboard and simple performance metrics (speed, accuracy, score).
    \item Exportable Godot project for Windows desktop.
\end{itemize}

\section*{Tools}
Godot (3.x/4.x), GDScript; optional Python utilities for word-list management. Assets: pixel sprites, WAV audio, simple tilemaps.

\section*{Deliverables}
Playable demo, source project (`.godot`), asset folder, README with build/export instructions, and short documentation on asset pipeline and optimization notes.

\section*{Timeline (2–4 weeks)}
Prototype (input + word manager): 1 week. Levels, assets, polish: 1–2 weeks. Testing and packaging: 1 week.

\section*{Expected Outcome}
A concise, well-documented demo that demonstrates practical 2D graphics workflows and delivers an enjoyable typing challenge.

\end{document}
